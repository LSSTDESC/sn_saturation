% 
% ======================================================================
\RequirePackage{desc-tex/styles/docswitch}
% \flag is set by the user, through the makefile:
%    make note
%    make apj
% etc.
\setjournal{\flag}

\documentclass[\docopts]{\docclass}

% You could also define the document class directly
%\documentclass[]{emulateapj}

% Custom commands from LSST DESC, see texmf/styles/lsstdesc_macros.sty
\usepackage{desc-tex/styles/lsstdesc_macros}

\usepackage{graphicx}
\graphicspath{{./}{./figures/}}
\bibliographystyle{desc-tex/bst/apj}

% Add your own macros here:



% 
% ======================================================================

\begin{document}

\title{The impact of LSST saturation on SN cosmology}

\maketitlepre

\begin{abstract}
We explore the impact of saturation on SN cosmology. In particular, we are interested in the effect of changing two equal timed snaps to a single snap in a visit. We find that the saturation effects are small at redshifts higher than 0.03.
\end{abstract}

% Keywords are ignored in the LSST DESC Note style:
\dockeys{latex: templates, papers: awesome}

\maketitlepost

% ----------------------------------------------------------------------
% 
\section{Introduction}
\label{sec:intro}
Survey strategy proposals frequently request that the default exposure time in LSST survey strategies
of $30~s$ per telescope visit or pointing be comprised of a single snap (or exposure), rather than the current two snaps of $15~s.$ While this proposal has advanatages, a disadvantage in doubling the exposure time is the collection of twice the number of photons during an exposure. For bright objects, such as nearby, near maximum supernovae, this could be more than the number of photon counts in t. This could happen for low redshift supernovae resulting in some of the light curve points being missing due to saturation. We quickly look at the impact of this effect on SNIa cosmology.  

% ----------------------------------------------------------------------
\section{Methods}
\label{sec:methods}
To study the impact of saturation, the starting point is the assumption that the full well potential is 100,000

We calculate the number of photons collected in a single pixel from the sky as well as from a point source. This is shown in \figref{fig:collected_photons}
\begin{itemize}
        \item We can calculate the number of photons collected (gain=1) from the magnitude in a band. This has been checked with using the lsst sims software. 
        \item This will be spread across a number of pixels. We use the approximation of a single Gaussian profile to model the PSF. The FWHM for this is known in terms of the standard deviation. We use this to calculate the density of photons at the brightest pixel, and multiply by the area to get the number of photons collected in the brightest pixel. 
        \item Given a sky brightness in mag/arc-sec, it is straightforward to calculate the number of photons from the sky collected. 
    \end{itemize}

Note this ignores other sources of photons such as a host galaxy. 


In the second case, we present the results from an SNAANA simulation based on the \verb minion_1016  opsim output. This includes contributions such as a host galaxy as well.
\section{Results}
\label{sec:results}
\begin{figure}
\includegraphics[width=0.9\columnwidth]{collected_photons}
\caption{Left: The number of sky photons per LSST pixel as a function of the sky brightness in units of $\mathrm{mag/arc-sec^2}$ in different bands of LSST. The ranges of $m_{sky}$ in different bands cover the range in OpSim (kraken\_2026 ). Right: The number of source photons in the brightest pixel for a source as a function of the source brightness in magnitude, for LSST bands g and r for minimum and median FWHM of 0.7\label{fig:collected_photons}}
\end{figure}
Figure~\ref{fig:collected_photons} shows that the number of sky photons collected in each pixel of the CCD is small compared to the assumed full well potential of 100000. This is high only for z because there are observations in z and y for very bright conditions. However, at the same sky brightness the redder bands contain far less photons. It also shows that in g band the source photons alone become a problem only when this is brighter than 16 mags. However source, sky and background galaxy photons could become a problem later. 

\begin{figure}
\includegraphics[width=0.9\columnwidth]{HubbleDiag}
\caption{The magnitude of a source with intrinsic magnitude of -19.3 mags\label{fig:hubble}}
\end{figure}
In \figref{hubble}, we look at the magnitudes of a source of absolute magnitude -19.3 including only the effect of cosmological dimming. Combinining these, it would seem like saturation is not a problem at redshifts > 0.03, with the effects starting to show between 0.02 and 0.03.

\begin{figure}
    \includegraphics[width=0.4\columnwidth]{nobs_saturate_gt0}
    \includegraphics[width=0.4\columnwidth]{nobs_saturate_gt1}
    \caption{\textbf{Left:}~Histogram of SNIa (black) in an SNANA simulation based on minion\_1016, along with a histogram of SNIa (filled red) where at least one epoch was saturated as a function of magnitude at peak (lower) and redshift (upper) for 2 snaps (left column) and 1 snap (right column). \textbf{Right} Same but the filled red histogram is of SN with more than one epoch saturated. \label{fig:saturation_hist}}
\end{figure}

In Figure~\ref{fig:saturation_hist}, we include further possible contributions to the photon count. While this is slightly different from the previous plots, it also suggests that there is little problem at redshifts above 0.03. 
\section{Discussion}
\label{sec:discussion}
We have heard elsewhere that a point source at LSST 17 mags might cause saturation. We can think about this later.

% ----------------------------------------------------------------------

\section{Conclusions}
\label{sec:conclusions}

Saturation, even with a single 30 sec snap per visit likely does to have a problem till around z of 0.02-0.03. SNIa at redshifts greater than 0.03 seem largely unaffected by saturation.

% ----------------------------------------------------------------------

\subsection*{Acknowledgments}

%\input{contributions} % Standard papers only: author contribution statements. For examples, see http://blogs.nature.com/nautilus/2007/11/post_12.html
% Standard papers only: A.B.C. acknowledges support from grant 1234 from ...

\input{desc-tex/ack/standard} % also available: key standard_short

\bibliography{main,desc-tex/bib/lsstdesc}

\end{document}
% ======================================================================
% 
